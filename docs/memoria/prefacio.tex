\cleardoublepage
\thispagestyle{empty}

\begin{center}
{\large\bfseries Agradecimientos}\\
\end{center}

A mi tutor, Rafa, y a mi mentor, Fernando, por haberme ayudado a profundizar mis conocimientos en el área del tratamiento de imágenes. A mis amigos, por haberme soportado, lo cual no es fácil, durante estos cuatro años. A mis padres, por haberme permitido continuar mis estudios y haberme apoyado en los momentos más difíciles. A mi abuela, quien por fin podrá cumplir su deseo de tener un nieto ingeniero.

\clearpage
\thispagestyle{empty}

\begin{center}
{\large\bfseries Segmentación de nadadores en piscinas con sistemas de captación de imágenes y vídeo}\\
\end{center}


\begin{center}
José Alberto Gómez García\\
\end{center}

%\vspace{0.7cm}
\noindent{\textbf{Palabras clave}: Detección de nadadores, Frecuencia de nado, Procesamiento de imágenes, Espacios de color, YCbCr, Sustracción de fondos, GSoC, Redes neuronales convolucionales, Aprendizaje profundo, YOLO}\\

\vspace{0.7cm}
\noindent{\textbf{Resumen}}\\

En este Trabajo de Fin de Grado, realizado con la colaboración de investigadores de la Facultad de Ciencias del Deporte de la Universidad de Granada, se propondrá un modelo para el cálculo de la frecuencia de nado media de un nadador a partir de una secuencia de vídeo. En primer lugar, describiremos el problema y estudiaremos las características del vídeo del que debemos extraer la información necesaria para los cálculos. Posteriormente, analizaremos el uso de técnicas clásicas del procesamiento de imágenes, así como técnicas basadas en aprendizaje profundo para detectar al nadador y compararemos los resultados obtenidos. Más tarde, utilizaremos la información que nos proporcionan estas técnicas para poder discriminar si se está realizando una brazada o no. A partir de estos datos, se calculará la frecuencia de nado media del nadador en cuestión.

Se observará que la aproximación basada en técnicas de aprendizaje profundo realiza detecciones más cercanas a la realidad, por lo que permite realizar predicciones de la frecuencia de nado media de mayor calidad. Finalmente, se propondrán diversas direcciones por las que se podría continuar el desarrollo de este trabajo en un futuro.

\clearpage


\thispagestyle{empty}

\begin{center}
{\large\bfseries Swimmers segmentation in swimming pools with image and video capture systems.}\\
\end{center}
\begin{center}
José Alberto Gómez García\\
\end{center}

%\vspace{0.7cm}
\noindent{\textbf{Keywords}: Swimmer detection, Swimming frequency, Image processing, Color spaces, YCbCr, Background subtraction, GSoC, Convolutional neural networks, Deep learning, YOLO}\\

\vspace{0.7cm}
\noindent{\textbf{Abstract}}\\


In this Bachelor's Degree Final Project we will propose a model for calculating the average swimming frequency of a swimmer given a video sequence. This work has been carried out with the collaboration of researchers from the Faculty of Sports Sciences of the University of Granada. First, we will describe the problem and study the key details of the video sequences from which we must extract the necessary information for the calculations. Then, we will analyze the use of classical image processing techniques and techniques based on deep learning to detect the swimmer and compare the results. Later, we will use the information provided by these techniques to be able to discriminate whether a stroke is being performed or not. From this data, the average swimming frequency of the swimmer will be calculated.

It will be observed that the approach based on deep learning techniques performs detections closer to reality, thus allowing higher-quality predictions of the average swimming frequency. Finally, several directions in which the development of this work could be continued in the future will be proposed.

\clearpage
\thispagestyle{empty}

\begin{comment}
\noindent\rule[-1ex]{\textwidth}{2pt}\\[4.5ex]

Yo, \textbf{José Alberto Gómez García}, alumno de la titulación Ingeniería Informática de la \textbf{Escuela Técnica Superior
de Ingenierías Informática y de Telecomunicación de la Universidad de Granada}, con DNI 26514779B, autorizo la
ubicación de la siguiente copia de mi Trabajo Fin de Grado en la biblioteca del centro para que pueda ser
consultada por las personas que lo deseen.

\vspace{6cm}

\noindent Fdo: José Alberto Gómez García

\vspace{2cm}

\begin{flushright}
Granada a 8 de julio de 2022.
\end{flushright}

\clearpage
\thispagestyle{empty}

\noindent\rule[-1ex]{\textwidth}{2pt}\\[4.5ex]

D. \textbf{Rafael Molina Soriano}, Profesor del Departamento de Ciencias de la Computación e Inteligencia Artificial de la Universidad de Granada.


\vspace{0.5cm}

\textbf{Informa:}

\vspace{0.5cm}

Que el presente trabajo, titulado \textit{\textbf{Segmentación de nadadores en piscinas con sistemas de captación de imágenes y vídeo}},
ha sido realizado bajo su supervisión por \textbf{José Alberto Gómez García}, y autoriza la defensa de dicho trabajo ante el tribunal
que corresponda.

\vspace{0.5cm}

Y para que conste, expide y firma el presente informe en Granada a 8 de julio de 2022.

\vspace{1cm}

\textbf{El director:}

\vspace{5cm}

\noindent \textbf{Rafael Molina Soriano}

\end{comment}

