\chapter{Introducción} \label{cap:capitulo1} 

La búsqueda de una forma física óptima mediante el entrenamiento deportivo ha tenido una gran importancia a lo largo de la historia del ser humano.
Ya en la antigüedad, durante la época helenística \cite{anta2013nuevas}, los atletas de élite que participaban en los juegos olímpicos se sometían a estrictos programas de entrenamiento, de un mínimo de diez meses de duración, y gozaban de un considerable reconocimiento social y económico. 

En 1896, con el resurgimiento de los juegos olímpicos modernos, los entrenadores se replantean los sistemas de entrenamientos a utilizar para alcanzar resultados óptimos \cite{anta2013nuevas}. Se pasa de un sistema de entrenamientos basado en el ensayo y error, a uno que busca técnicas y metodologías concretas que permitan maximizar las diferentes cualidades requeridas por los atletas de máximo nivel. Es a partir de este momento en el que la ciencia y la tecnología se ponen al servicio del deporte de élite con el objetivo de maximizar el rendimiento de los atletas.

Con el paso de los años, se introducen cada vez más sistemas cuyo objetivo es recabar información individual y personalizada para cada deportista. Hoy en día, tanto aficionados como profesionales, disponen de pulseras y relojes inteligentes que permiten medir el número de pasos andados, la velocidad a la que corremos, el esfuerzo realizado durante un partido de un tenis, la calidad de nuestro sueño, y otros muchos factores. Esta información proporciona una visión general sobre el desempeño deportivo y nuestra evolución conforme se entrena. 

Sin embargo, los deportistas de alto nivel requieren de sistemas más complejos que generen información concreta, detallada y específica para su ámbito deportivo. Por ejemplo, los velocistas emplean trajes especiales con multitud de sensores adheridos que permiten monitorizar el esfuerzo realizo por los músculos, la velocidad y aceleración del movimiento y demás parámetros, con los que se genera un modelo tridimensional del movimiento del deportista. A partir de la información generada, el entrenador puede sugerir rutinas de entrenamiento personalizadas que refuercen los puntos débiles del atleta, a la vez que se evitan lesiones \cite{rolecience}. También se emplean sistemas de realidad virtual \cite{sportvr} para aportar retroalimentación y entrenar correctamente la memoria gesticular, la cual resulta de especial importancia en deportes como el baloncesto, en los que movimientos rápidos y precisos son vitales.

Estos sistemas aportan una información de incalculable valor \cite{rolecience}. Consciente de esto, el profesor Raúl Arellano, catedrático de educación física de la Universidad de Granada especializado en natación, biomecánico, analista de la selección española de natación e investigador del grupo \textit{Aquatic Labs} de la UGR \cite{aquaticlabs}, ha utilizado multitud de sistemas de recogida de datos y ha analizado la influencia de diversos parámetros en el rendimiento de sus nadadores. 

En sus investigaciones \cite{raulvortice} \cite{raulonda} \cite{raulpropulsion},el profesor Raúl Arellano ha analizado, entre otros aspectos, la importancia del movimiento ondulatorio generado por el nadador en periodos subacuáticos y la correcta sincronización de las ondas generadas por brazo, tronco y piernas; ha caracterizado el vórtice generado en el agua por los mejores nadadores del mundo al introducirse en la piscina tras saltar del trampolín y ha intentado establecer correspondencias entre las fuerza generada por las piernas del nadador en el salto a la piscina y en la voltereta en la que se cambia el sentido del nado. Mediante este y otros trabajos busca perfeccionar los actuales métodos de entrenamiento y desarrollar nuevas metodologías que permitan maximizar el rendimiento de los deportistas a los que entrena. La importancia de su investigación trasciende el ámbito puramente científico y ha generado interés en la sociedad general, de hecho, su trabajo fue objeto de entrevista en la cadena de televisión andaluza Canal Sur en el año 2019. Esta puede ser visualizada en \cite{entrevistaraul}.

Con el objetivo de continuar su investigación, se ha desarrollado un modelo capaz de calcular la frecuencia media de nado de un nadador a partir de una secuencia de vídeo. En el presente trabajo detallaremos el proceso de desarrollo de la solución y su implementación. En el capítulo \ref{cap:capitulo2} describiremos el problema planteado con más detalle. En los capítulos siguientes, analizaremos y compararemos metodologías con las que detectar y rastrear al nadador durante su avance en la piscina. En concreto, en el capítulo \ref{cap:capitulo3} se analizan técnicas clásicas y en el capítulo \ref{cap:capitulo4} técnicas de aprendizaje profundo. Posteriormente, en el capítulo \ref{cap:capitulo5} analizaremos los datos recabados y ofreceremos un modelo del cálculo de la frecuencia media de nado. Finalmente, el capítulo \ref{cap:capitulo6} expondremos las conclusiones y discutiremos posibles lineas de trabajo futuro.